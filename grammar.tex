% TODOs: how to check LaTeX documentation inside of VS Code?
% how to use keyboard shortcuts, e.g CTRL+B -> textbf
\documentclass[12pt,a4paper]{article}

% -------------------------------------------------
% Language and encoding
% -------------------------------------------------
\usepackage[T1]{fontenc}
% \usepackage[utf8]{inputenc} not needed with modern LaTeX
\usepackage[slovak, german]{babel}

% -------------------------------------------------
% Packages
% -------------------------------------------------
\usepackage{xcolor} % needed to switch text color also in commands


% -------------------------------------------------
% Custom commands for language separation
% -------------------------------------------------
\newcommand{\DE}[1]{%
  \selectlanguage{german}#1\selectlanguage{slovak}
}

\newcommand{\SK}[1]{%
  \textcolor{gray}{\selectlanguage{slovak}#1}
}


\newenvironment{beispiel}
{%
  \par\smallskip
  \noindent\textbf{Beispiel:}
  \begin{list}{}{\leftmargin=1.5em}
  \item[]
}
{%
  \end{list}
  \par\smallskip
}


\title{Deutsche Grammatik}
\author{Juraj Paška}

\begin{document}

\maketitle
\tableofcontents


\section*{Die Einleitung}
Meine Notizen zur deutschen Grammatik. Die slovakische Übersetzung ist in Grau geschrieben. 
Beispiel: \SK{Toto je príklad slovenskej vety.} 

\section{Pasiv} 
Das Passiv bildet man mit werden + Partizip Perfekt. 
Beim Passiv ist die Aktion wichtiger als die Person.

\begin{beispiel}
    \begin{description}
        \item[Aktiv:] Der Mechaniker montiert den Motor. (Subjekt: der Mechaniker)
        \item[Passiv:] Der Motor wird montiert. (Subjekt: der Motor)
    \end{description}
  
\end{beispiel}

Das Akkusativ-Objekt aus dem Aktivsatz wird zum Subjekt im Passivsatz. Das Subjekt aus dem Aktivsatz nennt man meistens nicht. 
Man kann es mit \emph{von+Dativ} ergänzen \footnote{Grammatik Aktiv A1-B1.}. 



\end{document}
