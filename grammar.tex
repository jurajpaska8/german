% TODOs: how to check LaTeX documentation inside of VS Code?
% how to use keyboard shortcuts, e.g CTRL+B -> textbf
\documentclass[12pt,a4paper]{article}

% -------------------------------------------------
% Language and encoding
% -------------------------------------------------
\usepackage[T1]{fontenc}
% \usepackage[utf8]{inputenc} not needed with modern LaTeX
\usepackage[slovak, german]{babel}

% -------------------------------------------------
% Packages
% -------------------------------------------------
\usepackage{booktabs}
\usepackage{threeparttable} % for table notes
\usepackage[tableposition=top]{caption} % caption above tables with proper spacing
\usepackage{xcolor} % needed to switch text color also in commands
\usepackage{hyperref} % must go last
%\hypersetup{
%  colorlinks=true, % Enable colored links
%  linkcolor=blue,  % Color for internal links
%  citecolor=blue,  % Color for citations
%  urlcolor=blue    % Color for URLs
%}
% -------------------------------------------------
% Custom commands for language separation
% -------------------------------------------------
\newcommand{\DE}[1]{%
  \selectlanguage{german}#1\selectlanguage{slovak}
}

\newcommand{\SK}[1]{%
  \textcolor{gray}{\selectlanguage{slovak}#1}
}


\newenvironment{beispiel}[1][]
{%
  \par\smallskip
  \noindent\textbf{Beispiel\ifx&#1&\else: #1\fi}
  \begin{list}{}{\leftmargin=1.5em}
  \item[]
}
{%
  \end{list}
  \par\smallskip
}


\title{Deutsche Grammatik}
\author{Juraj Paška}

\begin{document}

\maketitle
\tableofcontents
\listoftables


\section*{Die Einleitung}
Meine Notizen zur deutschen Grammatik. Die slovakische Übersetzung ist in Grau geschrieben. 
Beispiel: \SK{Toto je príklad slovenskej vety.} 

\section{Passiv} 
Das Passiv bildet man mit werden + Partizip Perfekt. Beim Passiv ist die Aktion wichtiger als die Person.

\begin{beispiel}[Passiv 1]
    \begin{description}
        \item[Aktiv:] Der Mechaniker montiert den Motor. (Subjekt: der Mechaniker)
        \item[Passiv:] Der Motor wird montiert. (Subjekt: der Motor)
    \end{description}
\end{beispiel}
\begin{beispiel}[Passiv 2]
    \begin{description}
        \item[Aktiv:] Der Reiseleiter informiert die Gruppe. (Subjekt--Täter: der Reiseleiter)
        \item[Passiv:] Die Gruppe wird (vom Reiseleiter) informiert. (Subjekt: die Gruppe, Täter: der Reiseleiter)
    \end{description}
\end{beispiel}

In einem Passivsatz sind Subjekt und Täter verschieden. Der ,,Täter'' ist die Person, die die Aktion ausführt und kann weggelassen werden.
Das Akkusativ-Objekt aus dem Aktivsatz wird zum Subjekt im Passivsatz. Das Subjekt aus dem Aktivsatz nennt man meistens nicht. 
Man kann es mit \emph{von+Dativ} ergänzen\footnote{Grammatik Aktiv A1-B1.}. Deklinierung von \emph{werden} kann man in der Tabelle~\ref{tab:werden} nachschauen.

\begin{table}[htbp]
\centering
\caption{Konjugation des Verbs \emph{werden} im Präsens}
\label{tab:werden}
  \begin{tabular}{ll}
    \toprule
    \textbf{Person} & \textbf{Form von \emph{werden}} \\
    \midrule
    ich & werde \\
    du & wirst \\
    er / sie / es & wird \\
    wir & werden \\
    ihr & werdet \\
    sie / Sie & werden \\
    \bottomrule
  \end{tabular}  
\end{table}

Dort gibt es auch andere Zeitformen des Passivs. Diese werden mit den entsprechenden Zeitformen von \emph{werden} gebildet. 
Sehe Tabelle~\ref{tab:passivzeiten}.

\begin{table}[htbp]
\centering
\caption{Zeitformen des Passivs}
\label{tab:passivzeiten}
  \begin{threeparttable}
  \begin{tabular}{ll}
    \toprule
    \textbf{Zeitform} & \textbf{Beispiel} \\
    \midrule
    Präsens & Der Motor wird repariert. \\
    Präteritum & Der Motor wurde repariert. \\
    Perfekt & Der Motor ist repariert worden. \\
    Plusquamperfekt & Der Motor war repariert worden. \\
    Futur I* & Der Motor wird repariert werden. \\
    Futur II & Der Motor wird repariert worden sein. \\
    Konjuktiv 2 Gegenwart & Der Motor würde repariert. \\
    Konjuktiv 2 Vergangenheit & Der Motor wäre repariert worden. \\ 
    Konjuktiv 1 Gegenwart & Der Motor werde repariert. \\
    Konjuktiv 1 Vergangenheit & Der Motor sei repariert worden. \\ 
    \bottomrule
  \end{tabular}
  \begin{tablenotes}
  \footnotesize
  \item \emph{*} Diese Form wird nur sehr selten verwendet. 
  \item Quelle: \emph{Grammatik Aktiv B2-C1, 42}
  \end{tablenotes}
  \end{threeparttable}
\end{table}

\subsection{Passiv mit Modalverben}
\subsection{Vorgangspassiv, Zustandspassiv}

\section{Partizip 2}
Das Partizip 2 wird für die Bildung der zusammengesetzten Zeiten (Perfekt, Plusquamperfekt, Futur II) und des Passivs verwendet.
\SK{Je to príčastie minulé. Príčastie prítomne je Partizip 1 -- prechodník (tacnujúc - tanzend, spievajúc - singend)}

\subsection{Bildung von Partizip 2}
\textbf{Regelmäßige Verben}
\begin{itemize}
    \item Verben auf -en: ge + Stamm + t (machen - gemacht)
    \item Verben auf -eln/-ern: Stamm + t (handeln - gehandelt, ändern - geändert)
    \item Verben auf -ieren: Stamm + t (studieren - studiert)
    \item Verben auf -n: Stamm + en (atmen - geatmen)
\end{itemize}
\textbf{Unregelmäßige Verben}
\begin{itemize}
    \item Verben mit Vokalwechsel: ge + Stamm (mit Vokalwechsel) + en (fahren - gefahren, sehen - gesehen)
    \item Verben auf -inden: ge + Stamm + unden (finden - gefunden, schwinden - geschwunden)
    \item Verben auf -nehmen: ge + Stamm + ommen (nehmen - genommen, aufnehmen - aufgenommen)
    \item Verben auf -geben: ge + Stamm + eben (geben - gegeben, vergeben - vergeben)
    \item Verben auf -stehen: ge + Stamm + anden (stehen - gestanden, verstehen - verstanden)
\end{itemize}
\textbf{Untrennbare Verben} kein ge + Stamm + t (verstehen - verstanden)


\end{document}
